%%%%%%%%%%%%%%%%%%%%%%%%%%%%%%%%%%%%%%%%%%%%%%%%%%%%%%%%%%%%%%%%%%%%%%%%%%%%%%%%%%%%%%%%
% Criação de Fluxograma usando LaTeX
%
% Assunto: Fazer um jogo com duas escolhas, mas se voce errar a primeira delas, o jogo termina e voce perde, caso acerte o jogo continua e tem direito a escolha da segunda opcao, na qual, dependendo da
% escolha, voce pode ganhar ou perder.
%
% Autores:
%     Kauwane Fidelis de Souza
%     Thaísa Ribeiro Pimentel
%     Nome do Aluno 3
%
% Coordenação:
%     Prof. Dr. Ruben Carlo Benante
%
% Data: 2024-04-25
%%%%%%%%%%%%%%%%%%%%%%%%%%%%%%%%%%%%%%%%%%%%%%%%%%%%%%%%%%%%%%%%%%%%%%%%%%%%%%%%%%%%%%%%


%%%%%%%%%%%%%%%%%%%%%%%%%%%%%%%%%%%%%%%%%%%%%%%%%%%%%%%%%%%%%%%%%%%%%%%%%%%%%%%%%%%%%%%%
% Para gerar o PDF use o comando make com o makefile configurado:
%
%    $ make ext-programa2-benante-sobrenome1-sobrenome2.pdf
%
% O conteúdo do makefile é composto dos 3 seguintes comandos que ficam assim automatizados:
%    $ pdflatex exN-fluxograma.tex -o exN-fluxograma.pdf
%    $ bibtex biblio
%    $ pdflatex exN-fluxograma.tex -o exN-fluxograma.pdf


%%%%%%%%%%%%%%%%%%%%%%%%%%%%%%%%%%%%%%%%%%%%%%%%%%%%%%%%%%%%%%%%%%%%%%%%%%%%%%%%%%%%%%%%
% preambulo %%%%%%%%%%%%%%%%%%%%%%%%%%%%%%%%%%%%%%%%%%%%%%%%%%%%%%%%%%%%%%%%%%%%%%%%%%%%
\documentclass[a4paper,12pt]{article} %twocolumn
\usepackage[left=2.5cm,right=2cm,top=2.5cm,bottom=2cm]{geometry}
\usepackage[utf8]{inputenc} % letras acentuadas
\usepackage[portuguese]{babel} % tradução de títulos
\usepackage[colorlinks]{hyperref}
\usepackage{tikz} % para adicionar fluxogramas
\usepackage{algorithm} % ambiente para índice de algoritmos
\usepackage{algpseudocode} % fonte e estilo do algoritmo
\usepackage{graphicx} % permite adicionar imagens
\usepackage{indentfirst} % indenta o primeiro parágrafo também
\usepackage{url} % permite adicionar links de URLs e emails
% \usepackage{natbib}
%[noend]

\DeclareUrlCommand\email{\urlstyle{mm}} % comando para email bonito
\floatname{algorithm}{Algoritmo} % tradução da palavra algoritimo no ambiente de índice

\usetikzlibrary{shapes.geometric, shapes.symbols,arrows} % ajuste do tikz para incluir formas e setas

%%%%%%%%%%%%%%%%%%%%%%%%%%%%%%%%%%%%%%%%%%%%%%%%%%%%%%%%%%%%%%%%%%%%%%%%%%%%%%%%%%%%%%%%
% capa %%%%%%%%%%%%%%%%%%%%%%%%%%%%%%%%%%%%%%%%%%%%%%%%%%%%%%%%%%%%%%%%%%%%%%%%%%%%%%%%%
\title{Fluxograma: Princess Rescue}
\author{Kauwane Fidelis de Souza\\ Thaísa Ribeiro Pimente}

\begin{document}

\maketitle

%%%%%%%%%%%%%%%%%%%%%%%%%%%%%%%%%%%%%%%%%%%%%%%%%%%%%%%%%%%%%%%%%%%%%%%%%%%%%%%%%%%%%%%%
% definicao dos blocos do fluxograma (tikz) %%%%%%%%%%%%%%%%%%%%%%%%%%%%%%%%%%%%%%%%%%%%

\tikzstyle{line} = [draw, -latex']
\tikzstyle{startend} = [draw, ellipse,fill=red!20, minimum height=2em, node distance=1.55cm]
\tikzstyle{print} = [tape, fill=blue!20, draw, draw=black, minimum width=3cm, minimum height=1.4cm, text width=4.5em, text centered, tape bend top=none, tape bend height=0.2cm, node distance=1.55cm]
\tikzstyle{input} = [trapezium, trapezium left angle=60, trapezium right angle=90, minimum width=3cm, minimum height=1cm, text centered, draw=black, fill=blue!30, node distance=1.95cm]
\tikzstyle{process} = [rectangle, minimum width=3cm, minimum height=1cm, text centered, draw=black, fill=orange!30, node distance=1.55cm]

\tikzstyle{block} = [rectangle, draw, fill=blue!20, text width=5em, text centered, rounded corners, minimum height=4em, node distance=1.55cm]
\tikzstyle{decisionb} = [diamond, draw, fill=blue!20, text width=4.5em, text badly centered, inner sep=0pt, node distance=1.55cm]
\tikzstyle{decision} = [diamond, minimum width=3cm, minimum height=1cm, text centered, draw=black, fill=green!30, node distance=2.25cm]
\tikzstyle{empty} = [circle, fill=white, minimum width=0.01mm, node distance=2.55cm]

%%%%%%%%%%%%%%%%%%%%%%%%%%%%%%%%%%%%%%%%%%%%%%%%%%%%%%%%%%%%%%%%%%%%%%%%%%%%%%%%%%%%%%%%
% resumo %%%%%%%%%%%%%%%%%%%%%%%%%%%%%%%%%%%%%%%%%%%%%%%%%%%%%%%%%%%%%%%%%%%%%%%%%%%%%%%

\begin{abstract}

\textbf{Assunto:}  Programa Princess Rescue é um jogo texto de duas escolhas.

% descrever em poucas palavras seu projeto aqui

O programa tem interação com o usuário, onde ele assume o papel do Gato de Botas em uma jornada para salvar a princesa Kate.
O usuário deve tomar decisões difíceis e selecionar a arma correta para vencer o jogo com sucesso,
resultando em um final onde kate é salva e delvidada ao seu pai, o rei Alemão. Neste artigo iremos apresentar o seu fluxograma completo.
% e (opcionalmente) o seu algoritmo.

Após a modelagem do fluxograma e desenvolvimento da lógica de programação em algoritmo,
o programa será implementado na Linguagem de Programação \texttt{C}


\textbf{Local:} Escola Politécnica de Pernambuco - UPE/POLI

\textbf{Órgão Financiador:} N/A

\textbf{Caracterização:} Modelagem, Projeto e Implementação de Software em Linguagem \texttt{C}

% Este é o fim do resumo.

\end{abstract}


%%%%%%%%%%%%%%%%%%%%%%%%%%%%%%%%%%%%%%%%%%%%%%%%%%%%%%%%%%%%%%%%%%%%%%%%%%%%%%%%%%%%%%%%
% artigo %%%%%%%%%%%%%%%%%%%%%%%%%%%%%%%%%%%%%%%%%%%%%%%%%%%%%%%%%%%%%%%%%%%%%%%%%%%%%%%
% seção de introdução %%%%%%%%%%%%%%%%%%%%%%%%%%%%%%%%%%%%%%%%%%%%%%%%%%%%%%%%%%%%%%%%%%
\section{Introdução}

% Descrever melhor seu projeto aqui

Este programa tem intera¸c˜ao com o usualario e permite que escolha entre as alternati-
suas apresentações, dependendo de suas escolhas, o programa pode acabar se vocêˆe escolher
”errado”ou pode acabar de o usu´ario escolher fixer a escolha ”certa”.

O programa será modelado em \textit{fluxograma} em uma primeira fase, em seguida
sua lógica será desenvolvida em formato de \textit{algoritmo}, para então
na terceira fase ser implementado em Linguagem de Programação \texttt{C}.

Exemplo de letra em \textit{itálico é com textit}, e em \textbf{bold face é com textbf} e \texttt{tal tal tal é mono-espaçada type-writer}.

%%%%%%%%%%%%%%%%%%%%%%%%%%%%%%%%%%%%%%%%%%%%%%%%%%%%%%%%%%%%%%%%%%%%%%%%%%%%%%%%%%%%%%%%
% seção de objetivos %%%%%%%%%%%%%%%%%%%%%%%%%%%%%%%%%%%%%%%%%%%%%%%%%%%%%%%%%%%%%%%%%%%
\section{Fluxograma}

% adicionar aqui o fluxograma

\begin{tikzpicture}
    % colocar nodos
     \node (inicio) [startend] {Inicio};
     \node (txta) [print, below of=inicio, text width = 12cm, node distance=4cm] {Bem vinda(o) ao mundo de aventuras do PRINCES RESCUE!
Prepare-se para uma jornada repleta de escolhas difíceis e emoções, enquanto você, o destemido Gato de Botas, embarca em uma missão heróica para resgatar
a adorável Princesa Katy da Fortaleza das Maldições e devolve-la ao seu pai, o rei Alemão.};
     \node (inp1) [input, below of=txta, node distance=4cm, text width = 10cm] {Diante de você, dois caminhos e com cada um prometendo aventuras únicas e desafios emocionantes.
         Escolha com sabedoria e embarque em jornada épica: Bosque dos Jubilos ou Praia das Chamas.};
     \node (maior) [decision, below of=inp1, node distance=4cm] {Bosque dos Jubilos};
     \node (sim) [print, right of=maior, node distance=12cm, text width=12cm] {Oh, que tristeza! Você escolheu o destino errado antes mesmo de começar sua jornada. Você acabou
         escolhendo um caminho onde o Gato de Botas não tem condições de passar, ele não sabe nadar e você e acabou morrendo.};
                         \node (nao) [print, below of=maior, node distance=4cm, text width = 10cm] {Bom trabalho! Você escolheu com sabedoria, a trilha do Bosque dos Jubilos se abre diante de você, revelando um cenário incrível.};
       \node (fim) [startend, below of=nao, node distance=4cm] {Fim}
    % \node (vazio1) [empty, right of=fim, node distance=4cm] {};
    % Desenhar as setas
        \path [line] (inicio) -- (txta);
        \path [line] (txta) -- (inp1);
        \path [line] (inp1) -- (maior);
        \path [line] (maior) -- (sim);
        \path [line] (sim) -- (fim);
        \path [line] (maior) -- (nao);
        \path [line] (nao) -- (fim);
\end{tikzpicture}



\clearpage % inicia próxima seção em nova página
%%%%%%%%%%%%%%%%%%%%%%%%%%%%%%%%%%%%%%%%%%%%%%%%%%%%%%%%%%%%%%%%%%%%%%%%%%%%%%%%%%%%%%%%
% seção de justificativa %%%%%%%%%%%%%%%%%%%%%%%%%%%%%%%%%%%%%%%%%%%%%%%%%%%%%%%%%%%%%%%
% \section{Algoritmo}

% adicionar aqui o algoritmo (opcional)


% \clearpage % inicia próxima seção em nova página
%%%%%%%%%%%%%%%%%%%%%%%%%%%%%%%%%%%%%%%%%%%%%%%%%%%%%%%%%%%%%%%%%%%%%%%%%%%%%%%%%%%%%%%%
% Autores %%%%%%%%%%%%%%%%%%%%%%%%%%%%%%%%%%%%%%%%%%%%%%%%%%%%%%%%%%%%%%%%%%%%%%%%%%%%%%
\section*{Detalhamento dos Autores}

%%%%%%%%%%%%%%%%%%%%%%%%%%%%%%%%%%%%%%%%%%%%%%%%%%%%%%%%%%%%%%%%%%%%%%%%%%%%%%%%%%%%%%%%
% Discentes %%%%%%%%%%%%%%%%%%%%%%%%%%%%%%%%%%%%%%%%%%%%%%%%%%%%%%%%%%%%%%%%%%%%%%%%%%%%
\subsection*{Discentes}

\begin{enumerate}
    \item \textbf{Nome Completo:}Kauwane Fidelis De Souza
    \begin{description}
        \item [Email:] \email{kfs1@poli.br}
        \item [Endereço:Rua Boa luz, Nobre]
        \item [Matrícula:300890172]
        \item [CPF:136.779.154-57]
        \item [RG:10.870.221]
        \item [Telefone:81.9.9826-1971]
        \item [Currículo Lattes:] \url{http://lattes.cnpq.br/nnnnn}
    \end{description}

\item \textbf{Nome Completo:}Thaísa Ribeiro Pimentel
    \begin{description}
        \item [Email:] \email{trp@poli.br}
        \item [Endereço:]
        \item [Matrícula:]
        \item [CPF:]
        \item [RG:]
        \item [Telefone:]
        \item [Currículo Lattes:] \url{http://lattes.cnpq.br/nnnnn}
    \end{description}

    % \item \textbf{Nome Completo:} Fulano de tal
    % \begin{description}
        % \item [Email:] \email{blabla@poli.br}
        % \item [Endereço:]
        % \item [Matrícula:]
        % \item [CPF:]
        % \item [RG:]
        % \item [Telefone:]
        % \item [Currículo Lattes:] \url{http://lattes.cnpq.br/nnnnn}
    % \end{description}

%     \item \textbf{Nome Completo:} Fulano de tal
%     \begin{description}
%         \item [Email:] \email{blabla@poli.br}
%         \item [Endereço:]
%         \item [Matrícula:]
%         \item [CPF:]
%         \item [RG:]
%         \item [Telefone:]
%         \item [Currículo Lattes:] \url{http://lattes.cnpq.br/nnnnn}
%     \end{description}
\end{enumerate}


%%%%%%%%%%%%%%%%%%%%%%%%%%%%%%%%%%%%%%%%%%%%%%%%%%%%%%%%%%%%%%%%%%%%%%%%%%%%%%%%%%%%%%%%
% Docentes %%%%%%%%%%%%%%%%%%%%%%%%%%%%%%%%%%%%%%%%%%%%%%%%%%%%%%%%%%%%%%%%%%%%%%%%%%%%%
\subsection*{Docentes}

\begin{enumerate}
    \item \textbf{Nome Completo:}Ruben Carlo Benante
    \begin{description}
        \item [Email:] \email{rcb@upe.br}
        \item [Matrícula:] 11238-0
        \item [Currículo Lattes:] \url{http://lattes.cnpq.br/3366717378277623}
    \end{description}
\end{enumerate}


%%%%%%%%%%%%%%%%%%%%%%%%%%%%%%%%%%%%%%%%%%%%%%%%%%%%%%%%%%%%%%%%%%%%%%%%%%%%%%%%%%%%%%%%
% referências bibliográficas %%%%%%%%%%%%%%%%%%%%%%%%%%%%%%%%%%%%%%%%%%%%%%%%%%%%%%%%%%%
%\section*{Referências Bibliográficas}

% cite todos, mesmo os não referenciados %%%%%%%%%%%%%%%%%%%%%%%%%%%%%%%%%%%%%%%%%%%%%%%
\nocite{*}


%%%%%%%%%%%%%%%%%%%%%%%%%%%%%%%%%%%%%%%%%%%%%%%%%%%%%%%%%%%%%%%%%%%%%%%%%%%%%%%%%%%%%%%%
% se necessario %%%%%%%%%%%%%%%%%%%%%%%%%%%%%%%%%%%%%%%%%%%%%%%%%%%%%%
% troca autor and autor por autor & autor, na bibliografia. O dcu usa "and"
%\renewcommand{\harvardand}{\&} % troca and pro &. O dcu usa "and"

% Estilos de bibliografia %%%%%%%%%%%%%%%%%%%%%%%%%%%%%%%%%%%%%%%%%%%%%%%%%%%%%%%%%%%%%%
% \bibliographystyle{abnt-alf} % Estilo alfabético da ABNT. Opção [num] para estilo numérico
% \bibliographystyle{apalike}
% \bibliographystyle{dcu} %citacao como (autor and autor, ano). Parece apalike. Rev. Control. Automacao. Use com harvard
% \bibliographystyle{agsm} % padrao harvard fica (autor & autor ano).
\bibliographystyle{acm}

%%%%%%%%%%%%%%%%%%%%%%%%%%%%%%%%%%%%%%%%%%%%%%%%%%%%%%%%%%%%%%%%%%%%%%%%%%%%%%%%%%%%%%%%
% arquivo de banco de dados das referências %%%%%%%%%%%%%%%%%%%%%%%%%%%%%%%%%%%%%%%%%%%%
% renomear para o número do exercício correto
% o arquivo de bibliografia pode se chamar qualquer coisa, isso não muda o comando de gerar o PDF.
% Por exemplo para 'mybiblio.bib', use \bibliography{mybiblio} e os comandos pdflatex e bibtex continuam os mesmos identicos com exN.
\bibliography{biblio}

\end{document}
